\documentclass[aps,twocolumn,secnumarabic,nobalancelastpage,amsmath,amssymb,
nofootinbib,superscriptaddress]{revtex4-1}


\usepackage{graphics}       % standard graphics specifications
\usepackage{graphicx}       % alternative graphics specifications
\usepackage{longtable}      % helps with long table options
\usepackage{url}            % for on-line citations
\usepackage{bm}             % special 'bold-math' package
\usepackage[ngerman]{babel} % deutsche Siblentrennung
\usepackage[utf8]{inputenc} % Umlaute
dd
\def\andname{\hspace*{-0.5em},} % definiert die Trennung zwischen 2 Autoren neu

% Titelseite
\begin{document}
\title{F-Praktikum: Luminescence Spectrocopy}
\author         {Ch. Egerland}
\email[Email: ]{egerlanc@physik.hu-berlin.de}
\author         {M. Pfeifer}
\email[Email: ]{mpfeifer@physik.hu-berlin.de}
\affiliation    {Humboldt-Universität zu Berlin, Institut für Physik}
\date[Versuchsdatum: ]{06.07.2017}

%%%%%%%%%%%%%%%%%%%%%%%%%%%%%%%%%%%%%%%%%%%%%%%%%%%%%%%%%%%%%%%%%%%%%%%%%%%%%%%%
\begin{abstract}
Untersucht wird das Lumineszenzspektrum einer InGaP-Photodiode im Temperaturbereich zwischen
$T=80\text{ K}$ und $T=250\text{ K}$. Die Fluoreszenzmessung wird mithilfe eines Czerny-Turner-Spektrometers
und einer Photomultiplier-Tube realisiert. Wir treffen Aussagen über den Verlauf des Spektrums, die
Temperaturabhängigkeit des Lumineszenzpeaks, den thermischen Einluss auf die integrierte Peakintensität und
wir geben eine Abschätzung für die Aktivierungsenergie von InGaP an.
\end{abstract}


\maketitle


%%%%%%%%%%%%%%%%%%%%%%%%%%%%%%%%%%%%%%%%%%%%%%%%%%%%%%%%%%%%%%%%%%%%%%%%%%%%%%%%
\section{Wie wir in Zukunft Paper schreiben}


Wir untergliedern in:
\begin{enumerate}
\item Theorie: kleine Zusammenfassung der Theorie, die hinter'm Experiment steht

\item Experiment: Aufbau, Funktionsweise

\item Daten und Analyse: Auswertung

\item Schlussfolgerung:

\end{enumerate}

{\bf So und nicht anders!}
\\

Lorem ipsum dolor sit amet, consectetur adipiscing elit. Nam id facilisis ligula,
a ultrices nibh. Nullam suscipit tellus nec mauris fermentum, ornare luctus neque
tincidunt. Aenean commodo tincidunt varius. Phasellus faucibus metus non erat
consectetur bibendum. Duis et luctus risus, at egestas justo. Nunc eleifend lacus
ac laoreet scelerisque. Aenean cursus dignissim magna in ultrices. In eget nisl
quis nisi.


%%%%%%%%%%%%%%%%%%%%%%%%%%%%%%%%%%%%%%%%%%%%%%%%%%%%%%%%%%%%%%%%%%%%%%%%%%%%%%%%

\section{Theorie}

Unter Lumineszenz versteht man Strahlung, die beim Übergang eines Systems von einem angeregten Zustand
in einen niederenergetischen Zustand emittiert wird. Bei Halbleitern finden diese elektronischen Übergänge vor allem
vom Leitungsband ins Valenzband statt, so dass bei Anregung des Materials mit ausreichend Energie ($E>E_g$) Lumineszenz-
photonen mit einer der Bandlücke entsprechenden Frequenz erzeugt werden. Liegt ein dotierter Halbleiter vor, speziell z.B.
eine Heterostruktur mit pn-Übergang, gibt es im Bereich zwischen Leitungs- und Valenzband weiterhin Donator- (knapp unter
Leitungsbandkante) und Akzeptorzustände (knapp über der Valenzbandkante). Ebenfalls für die Lumineszenz relevant sind sog.
Exzitonen-Zustände (gebundene Elektron-Loch-Paare). Bei guter Kühlung (wenig thermischer Anregung) sind neben der Fluoreszenz
aufgrund der Rekombination von Elektronen aus dem Valenzband ins Leitungsband auch andere Übergänge, z.B. vom Donator- zum
zum Akzeptorniveau (D,A) oder ins Valenzband (D,h) sichtbar.

Hier mal noch Beispiele wie man Gleichungen richtig referenziert, siehe
Gleichung.~\ref{eq:first-equation} :
\begin{equation}
   \chi_+(p)\alt{\bf [}2|{\bf p}|(|{\bf p}|+p_z){\bf ]}^{-1/2}
   \left(
   \begin{array}{c}
      |{\bf p}|+p_z\\
      px+ip_y
   \end{array}\right)
   \label{eq:first-equation}
\end{equation}


Oder auch in der Zeile: $\vec{\psi_1} = |\psi_1\rangle \equiv c_0|0\rangle +
c_1|1\rangle \chi^2 \approx
\prod\sum\left[\frac{y_i-f(x_i)}{\sigma_i}\right]^2 |\psi_1\rangle
\sim \lim_{\mu \rightarrow \infty}p(x;\mu) \geq \frac{1}{\sqrt{2 \pi
\mu}} e^{-(x-\mu)^2 / 2\mu}P(x) \ll \int_{-\infty}^x p(x')dx'a
\times b \pm c \Rightarrow \nabla \hbar$.

Manchmal auch über mehr als eine Zeile, siehe Equation~\ref{eq:multilineeq}:
\begin{eqnarray}
  \sum \vert M^{\text{viol}}_g \vert ^2
   &=&  g^{2n-4}_S(Q^2)~N^{n-2} (N^2-1)
\nonumber
\\
   &&   \times \left( \sum_{i<j}\right) \sum_{\text{perm}}
            \frac{1}{S_{12}}  \frac{1}{S_{12}} \sum_\tau c^f_\tau
\,.
\label{eq:multilineeq}
\end{eqnarray}

Natürlich gibts auch die guten alten subequations wie (\ref{subeq:1}) und
(\ref{subeq:2}):
\begin{subequations}
\label{eq:whole}
\begin{equation}
  \left\{
      abc123456abcdef\alpha\beta\gamma\delta1234556\alpha\beta
       \frac{1\sum^{a}_{b}}{A^2}
  \right\}
%
\,\label{subeq:1}
\end{equation}
\begin{eqnarray}
  {\cal M} &=& ig_Z^2(4E_1E_2)^{1/2}(l_i^2)^{-1}
                (g_{\sigma_2}^e)^2\chi_{-\sigma_2}(p_2)
\nonumber\\
  &&\times [\epsilon_i]_{\sigma_1}\chi_{\sigma_1}(p_1).\label{subeq:2}
\end{eqnarray}
\end{subequations}

Lorem ipsum dolor sit amet, consectetur adipiscing elit. Nam id facilisis ligula,
a ultrices nibh. Nullam suscipit tellus nec mauris fermentum, ornare luctus neque
tincidunt. Aenean commodo tincidunt varius. Phasellus faucibus metus non erat
consectetur bibendum. Duis et luctus risus, at egestas justo. Nunc eleifend lacus
ac laoreet scelerisque. Aenean cursus dignissim magna in ultrices. In eget nisl
quis nisi.


%%%%%%%%%%%%%%%%%%%%%%%%%%%%%%%%%%%%%%%%%%%%%%%%%%%%%%%%%%%%%%%%%%%%%%%%%%%%%%%%
\section{Experiment}

Text zum Experiment mit vielleicht einer Grafik:

\begin{figure}[h]

\caption{ein Bild vom Versuchsaufbau, Aus: \cite{melissinos1966,melissinos2003}.}
\label{fig:samplefig}
\end{figure}

Lorem ipsum dolor sit amet, consectetur adipiscing elit. Nam id facilisis ligula,
a ultrices nibh. Nullam suscipit tellus nec mauris fermentum, ornare luctus neque
tincidunt. Aenean commodo tincidunt varius. Phasellus faucibus metus non erat
consectetur bibendum. Duis et luctus risus, at egestas justo. Nunc eleifend lacus
ac laoreet scelerisque. Aenean cursus dignissim magna in ultrices. In eget nisl
quis nisi.



%%%%%%%%%%%%%%%%%%%%%%%%%%%%%%%%%%%%%%%%%%%%%%%%%%%%%%%%%%%%%%%%%%%%%%%%%%%%%%%%
\section{Daten und Analyse}

Lorem ipsum dolor sit amet, consectetur adipiscing elit. Nam id facilisis ligula,
a ultrices nibh. Nullam suscipit tellus nec mauris fermentum, ornare luctus neque
tincidunt. Aenean commodo tincidunt varius. Phasellus faucibus metus non erat
consectetur bibendum. Duis et luctus risus, at egestas justo. Nunc eleifend lacus
ac laoreet scelerisque. Aenean cursus dignissim magna in ultrices. In eget nisl
quis nisi. Tabelle \ref{tab:table1}:


\begin{table}[h]
\caption{\label{tab:table1}Eine Tabelle mit Fußnoten}
\begin{ruledtabular}
\begin{tabular}{cccccccc}
 &$r_c$ (\AA)&$r_0$ (\AA)&$\kappa r_0$&
 &$r_c$ (\AA) &$r_0$ (\AA)&$\kappa r_0$\\
\hline
Cu& 0.800 & 14.10 & 2.550 &Sn\footnotemark[1] & 0.680 & 1.870 & 3.700 \\
Ag& 0.990 & 15.90 & 2.710 &Pb\footnotemark[1] & 0.450 & 1.930 & 3.760 \\
Tl& 0.480 & 18.90 & 3.550 & & & & \\
\end{tabular}
\end{ruledtabular}
\footnotetext[1]{Entnommen aus Ref.~\cite{bevington2003}.}
\end{table}




%%%%%%%%%%%%%%%%%%%%%%%%%%%%%%%%%%%%%%%%%%%%%%%%%%%%%%%%%%%%%%%%%%%%%%%%%%%%%%%%
\section{Schlussfolgerung}

Schlussoflgerung, sollten wir mal was von nem Buch oder so entnehmen nutzen wir:


\begin{quote}
  Ein Zitat mit Referenz auf das Buch\cite{melissinos1966}
\end{quote}

Lorem ipsum dolor sit amet, consectetur adipiscing elit. Nam id facilisis ligula,
a ultrices nibh. Nullam suscipit tellus nec mauris fermentum, ornare luctus neque
tincidunt. Aenean commodo tincidunt varius. Phasellus faucibus metus non erat
consectetur bibendum. Duis et luctus risus, at egestas justo. Nunc eleifend lacus
ac laoreet scelerisque. Aenean cursus dignissim magna in ultrices. In eget nisl
quis nisi.


%%%%%%%%%%%%%%%%%%%%%%%%%%%%%%%%%%%%%%%%%%%%%%%%%%%%%%%%%%%%%%%%%%%%%%%%%%%%%%%%
\bibliography{sample-paper}
\bibliographystyle{prsty}
\begin{thebibliography}{99}
\bibitem{abkuerzung1}Autor, Titel, Verlag,  [1945]
\bibitem{abkuerzung2}Autor, Titel, Verlag,  [1945]
\bibitem{abkuerzung3}Autor, Titel, Verlag,  [1945]
\bibitem{abkuerzung4}Autor, Titel, Verlag,  [1945]
\end{thebibliography}


%%%%%%%%%%%%%%%%%%%%%%%%%%%%%%%%%%%%%%%%%%%%%%%%%%%%%%%%%%%%%%%%%%%%%%%%%%%%%%%%
\clearpage
\appendix

\section{Sonstiges}
Hier sehen wir einen Beispiel Anhang und so könnte man Code in Latex einbinden:
\begin{verbatim}
> mkdir ~/8.13
> mkdir ~/8.13/papers
> mkdir ~/8.13/papers/template
> cd ~/8.13/papers/template
\end{verbatim}


%%%%%%%%%%%%%%%%%%%%%%%%%%%%%%%%%%%%%%%%%%%%%%%%%%%%%%%%%%%%%%%%%%%%%%%%%%%%%%%%


\end{document}
